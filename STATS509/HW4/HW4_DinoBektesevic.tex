\documentclass{article}
\usepackage{amsmath,amsfonts,amsthm,amssymb,amsopn,bm}
\usepackage[margin=.9in]{geometry}
\usepackage{graphicx}
\usepackage{url}
\usepackage[usenames,dvipsnames]{color}
\usepackage{fancyhdr}
\usepackage{multirow}
\usepackage{listings}
\usepackage{hyperref}

\definecolor{keywords}{RGB}{255,0,90}
\definecolor{comments}{RGB}{0,0,113}
\definecolor{red}{RGB}{160,0,0}
\definecolor{green}{RGB}{0,150,0}
 
\lstset{language=Python, 
        basicstyle=\ttfamily\tiny, 
        keywordstyle=\color{keywords},
        commentstyle=\color{comments},
        stringstyle=\color{red},
        showstringspaces=false}

\newcommand{\field}[1]{\mathbb{#1}}
\newcommand{\1}{\mathbf{1}}
\newcommand{\E}{\mathbb{E}} 
\newcommand{\Z}{\mathbb{Z}} 
\renewcommand{\P}{\mathbb{P}}
\newcommand{\R}{\field{R}} % real domain
% \newcommand{\C}{\field{C}} % complex domain
\newcommand{\F}{\field{F}} % functional domain
\newcommand{\T}{^{\textrm T}} % transpose
\def\diag{\text{diag}}

%% operator in linear algebra, functional analysis
\newcommand{\inner}[2]{#1\cdot #2}
\newcommand{\norm}[1]{\left\|#1\right\|}
\newcommand{\twonorm}[1]{\|#1\|_2^2}
% operator in functios, maps such as M: domain1 --> domain 2
\newcommand{\Map}[1]{\mathcal{#1}}
\renewcommand{\theenumi}{\alph{enumi}} 

\newcommand{\Perp}{\perp \! \! \! \perp}

\newcommand\independent{\protect\mathpalette{\protect\independenT}{\perp}}
\def\independenT#1#2{\mathrel{\rlap{$#1#2$}\mkern2mu{#1#2}}}
\newcommand{\vct}[1]{\boldsymbol{#1}} % vector
\newcommand{\mat}[1]{\boldsymbol{#1}} % matrix
\newcommand{\cst}[1]{\mathsf{#1}} % constant
\newcommand{\ProbOpr}[1]{\mathbb{#1}}
\newcommand{\points}[1]{\small\textcolor{magenta}{\emph{[#1 points]}} \normalsize}
\date{{}}

\setlength\parindent{0px}

\begin{document}
\title{Homework \#4}
\author{\normalsize{Winter 2020, STATS 509}\\
\normalsize{Dino Bektesevic}}
\maketitle

\section*{Problem 1}
Suppose that $X$ is a positive random variable so that 
$P(X > 0) = 1$. Prove that for any $t<0$, $M_X(t) = E[e^{tX}] < \infty$.

\emph{Hint: Find a relationship between $e^{tX}$ and $1$, for $t < 0$.
Use the fact that if for all $x$, $g(x) \leq h(x)$ then $\int g(x) dx \leq \int h(x)dx$.}

\begin{align*}
    M_X(t)  &= E[e^{tX}] = \int_{-\infty}^\infty e^{tx} f_X(x) dx \\
    &= \int_0^\infty e^{tx} f_X(x) dx
\end{align*}

For $t<0$:
$$ e^{tx} f_X(x) \leq f_X(x) \forall f_X(x)$$
because $e^{kx}\leq 1 \forall k<0$ so it follows:
\begin{align*}
    \int_0^\infty e^{tx} f_X(x) dx &\leq \int_0^\infty f_X(x) dx \\
     \int_0^\infty e^{tx} f_X(x) dx &\leq P(X>0) \\
      \int_0^\infty e^{tx} f_X(x) dx &\leq 1
\end{align*}




\newpage
\section*{Problem 2.}
Let $X$ be an exponential random variable with parameter $\lambda > 0$.
\begin{enumerate}
    \item Use Chebyshev's inequality to find an upper bound on $Pr(|X- E[X] | \geq k)$

    \begin{align*}
        P(|X-c|) \geq d) &\leq \frac{E[(X-c)^2]}{d^2} \\
        P(|X-\mu|) \geq k) &\leq \frac{E[(X-\mu)^2]}{k^2} \\
        P(|X-\mu|) \geq k) &\leq \frac{V[X]}{k^2} \\
        P(|X-\mu|) \geq k) &\leq \left(\frac{1}{k\lambda}\right)^2
    \end{align*}

    \item Calculate $E[ |X - E[X]|]$ and then use Markov's inequality to find a different upper bound on $Pr(|X- E[X] | \geq k)$.\par
    {\it Hint: first remove the absolute value by splitting the integral into two pieces. Each piece can then be integrated using the Exponential CDF (or just integrating directly) and integration by parts.}
    
    \begin{align*}
        E[|X - E[X]|] &= E[|X - \mu|] = \int_{-\infty}^\infty |x-\mu|f_X(x) dx \\
        &= \int_{0}^\infty |x-\mu|\lambda e^{-\lambda x} dx \\
        &= \int_0^{1/\lambda} \left(\frac{1}{\lambda} - x\right)\lambda e^{-\lambda x} dx 
            + \int_{1/\lambda}^\infty \left(x - \frac{1}{\lambda} \right)\lambda e^{-\lambda x} dx \\
        &= \int_0^{1/\lambda} e^{-\lambda x} dx 
            - \int_0^{1/\lambda} \lambda xe^{-\lambda x} dx
            + \int_{1/\lambda}^\infty \lambda  xe^{-\lambda x} dx 
            - \int_{1/\lambda}^\infty e^{-\lambda x} dx \\
        &= -\frac{e^{-\lambda x}}{\lambda} \bigg|_0^{1/\lambda} 
            + \lambda \frac{\lambda x+1}{\lambda^2}e^{-\lambda x} \bigg|_0^{1/\lambda}
            - \lambda \frac{\lambda x+1}{\lambda^2}e^{-\lambda x} \bigg|_{1/\lambda}^\infty 
            + \frac{e^{-\lambda x}}{\lambda} \bigg|_{1/\lambda}^\infty \\
        &= -\frac{e^{-\lambda x}}{\lambda} \bigg|_0^{1/\lambda} 
            + \lambda \frac{\lambda x+1}{\lambda^2}e^{-\lambda x} \bigg|_0^{1/\lambda}
            - \lambda \left(\frac{\lambda x}{\lambda^2}e^{-\lambda x} + \frac{e^{-\lambda x}}{\lambda^2} \right)\bigg|_{1/\lambda}^\infty 
            + \frac{e^{-\lambda x}}{\lambda} \bigg|_{1/\lambda}^\infty \\
        &= \left(-\frac{1}{e\lambda}+ \frac{1}{\lambda}\right) 
            + \lambda \left(\frac{2}{e\lambda^2} - \frac{1}{\lambda^2} \right) 
            - \lambda \left(0 + 0 - \frac{2}{e\lambda^2} - \frac{1}{\lambda^2}\right) 
            + \left(0 - \frac{1}{e\lambda}\right) \\
        &= -\frac{1}{e\lambda} + \frac{1}{\lambda} 
           + \frac{2}{e\lambda} - \frac{1}{\lambda}  
           + \frac{2}{e\lambda} + \frac{1}{\lambda} 
           - \frac{1}{e\lambda}\\
        &= \frac{2}{e\lambda} + \frac{1}{\lambda} \\
        &= \frac{4-2\lambda}{e\lambda^2}
    \end{align*}
    Markov's inequality is then:
    $$P(|X-\mu|\geq k) = \frac{e-2}{e \lambda k}$$

    \item[(c)] For which values of $k$ is the bound in (b) lower than that given in (a)?
    
    \begin{align*}
        1 &> \frac{P_b}{P_a}\\
        1 &>\frac{(e-2)\lambda^2 k^2}{e\lambda k} \\ %\frac{\frac{e-2}{e\lambda k}}{\frac{1}{\lambda^2 k^2}}
        1 &> \frac{e-2}{e}\lambda k \\
        k &< \frac{e}{\lambda (e-2)} \approx \frac{3.784}{\lambda}
    \end{align*}
\end{enumerate}



\newpage
\section*{Problem 3} 
If the joint probability density of the two random variables $X$ and $Y$ is given by:
$$ f_{XY}(x,y) =\left\{
    \begin{array}{ll}
        2 & \hbox{for } 0 < x < y <1 \\
        0 & \hbox{otherwise.}
    \end{array}
\right.$$

\begin{enumerate}
    \item Find $P(Y<0.5)$. {\it Hint: Draw a graph.}  

    \begin{align*}
        P(Y<0.5) &= \int_0^1 \int_0^y f(x,y)dxdy \\
        &= \int_0^1 \int_0^y 2 dxdy = 2 \frac{y^2}{2}\bigg |_0^1 \\
        &= 1
    \end{align*}


    \item Find $P(X<0.5)$. 
    
    \begin{align*}
        P(X<0.5) &= \int_0^{0.5}\int_x^1 f(x,y)dydx \\
        &= \int_0^1 (2-x) dx = 2x|_0^{0.5} -  \frac{x^2}{2}\bigg |_0^{0.5} \\
        &= 1 - \frac{1}{8} = \frac{7}{8}
    \end{align*}
    
    
    \item Find $P(X+Y < 1)$ 
    
    \begin{align*}
        P(X+Y<1) &= \int_0^{0.5}\int_x^{1-x} f(x,y)dydx \\
        &= 2\int_0^{0.5} (1-2x) dx = 2\left(x|_0^{0.5} -  2\frac{x^2}{2}\bigg |_0^{0.5}\right) \\
        &= 2\frac{1}{4} = \frac{1}{2}
    \end{align*}
        
    
    \item Find the joint CDF $F(x,y)$. {\it (You will need to consider several cases.)}  
    $$F(x,y) = \begin{cases} 
	        0   &\mbox{if } x < 0 \mbox{ or } y<0   \\
	        0   &\mbox{if } y > 1 \mbox{ or } x > y \\
            2xy &\mbox{if } y < 1 \mbox{ and } x<y
        \end{cases}  
    $$
\end{enumerate}




\newpage
\section*{Problem 4} 
Suppose that $X$ and $Y$ are continuous random variables with the following joint cdf:

\begin{equation}\label{eq:one}
F_{XY}(x,y) = \widetilde{F}(x)\widetilde{F}(y)
\end{equation}

where $\widetilde{F}(\cdot)$ is a (univariate) cdf. 

{{(Aside)} \it It follows from {\rm (\ref{eq:one})} that $X$ and $Y$ are independent random variables with the same
marginal distribution.} 

\begin{itemize}
    \item[(a)] For a fixed value $t$, express $P(X \leq t, Y \leq t)$ in terms of $\widetilde{F}(\cdot)$.\par
    {\it Hint: Recall the definition of  a joint cdf $F_{XY}(x,y)$.}
    
    By definition $F(x,y)= P(X\leq x, Y\leq y)$ so it follows that:
    $$P(X \leq t, y\leq t) = F_{XY}(t,t) = \Tilde{F}(t)\Tilde{F}(t) = \Tilde{F}^2(t)$$
    
    \item[(b)] Let $T= \max (X,Y)$. Using your answer to (a) find the cdf $F_T(t)$ and the pdf $f_T(t)$ for $T$:\par
    {\it Hint: If $X\leq t$ {\bf and} $Y\leq t$, what can one say about $\max(X,Y)$}?
    
    Because $T= \max (X,Y)$, for $T\leq t$ to hold both $X\leq t$ and $Y\leq t$ have to be satisfied:
    
    By definition:
    \begin{align*}
        f_T(t) = \frac{\partial}{\partial t} F_T(t) &=  \frac{\partial}{\partial t} \Tilde{F}^2(t) \\
        &= 2\Tilde{F}(t)\Tilde{f(t)}
    \end{align*}

\end{itemize}




\newpage
\section*{Problem 5}  If the joint CDF for two random variables, $X$ and $Y$, is  
given by:
\[
F_{XY}(x,y) =\left\{\begin{array}{ll}
(1-e^{-x^2})(1-e^{-y^2}) & \hbox{for } x>0, y>0\\[4pt]
0 & \hbox{otherwise}
\end{array}
\right.
\]
Find the joint pdf $f_{XY}(x,y)$.



\newpage
\section*{Problem 6}  Let $X$ and $Y$ have the joint density function (here $k>0$ is an unknown constant):
\[
f_{XY}(x,y) = \left\{\begin{array}{cc}
k(x-y), & 0\leq y\leq x \leq 1;\\[4pt]
0 & \hbox{otherwise,}
\end{array}
\right.
\]

\begin{itemize}
\item[(a)] Sketch the support of the joint density. You will need to use this in answering the next three question parts.
\item[(b)] Find the constant of proportionality $k$.
\item[(c)] Find the marginal densities $f_X(x)$ and $f_Y(y)$.
\item[(d)] Find the conditional densities of $f_{Y|X}(y\mid x)$ and $f_{X|Y}(x\mid y)$.\par
 Remember to carefully state for which values of the conditioning variable these are defined.
\end{itemize}

\eject




\newpage
\section*{Problem 7} 
If the joint probability density of the two random variables $X$ and $Y$ is given by:
\[
f_{XY}(x,y) =\left\{\begin{array}{ll}
4xy & \hbox{for } 0 < y<1,\; 0<x <1,\\
0 & \hbox{otherwise.}
\end{array}
\right.
\]
Let $D=(X-Y)$ and $S=(X+Y)$.  
\begin{itemize}
\item[(a)] Solve for $X$  and $Y$ in terms of $D$ and $S$.

\item [(b)] Find the support of $f_{DS}(d,s)$ and draw a sketch.

\item[(c)] Find the joint pdf $f_{DS}(d,s)$.

\item[(d)] Find the marginal densities $f_D(d)$ and $f_S(s)$.

\item[(e)] Find the conditional densities of $f_{D|S}(d\mid s)$ and $f_{S|D}(s\mid d)$.\par

 Remember to carefully state for which values of the conditioning variable these are defined.
\end{itemize}



\newpage
\section*{Problem 8} The random variables $(X,Y)$ have the following joint pdf:
\[
f_{XY}(x,y) =\left\{\begin{array}{ll}
2y & \hbox{for } 0 < x < 1 \hbox{ and } 0 < y <1,\\ 
0 & \hbox{otherwise.}
\end{array}
\right.
\]
Consider the transformation: $U = XY$ and $ V=X$.

\begin{itemize}
\item[(a)] Draw the support of the joint density $f_{XY}(x,y)$ for $(X,Y)$.

\item[(b)] Solve for $X$ and $Y$ in terms of $U$ and $V$.

\item[(c)] Carefully find the support for the joint density for $(U,V)$.
 Then find the joint density $f_{UV}(u,v)$, remember to include the support.
\item[(d)] Find the marginal densities $f_U(u)$ for $U$ and  $f_V(v)$ for $V$.


\end{itemize}


\end{document}
